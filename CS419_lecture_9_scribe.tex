%
% This is the LaTeX template file for lecture notes for CS294-8,
% Computational Biology for Computer Scientists.  When preparing 
% LaTeX notes for this class, please use this template.
%
% To familiarize yourself with this template, the body contains
% some examples of its use.  Look them over.  Then you can
% run LaTeX on this file.  After you have LaTeXed this file then
% you can look over the result either by printing it out with
% dvips or using xdvi.
%
% This template is based on the template for Prof. Sinclair's CS 270.

\documentclass[11pt, twosides]{article}
\usepackage[utf8]{inputenc}
\usepackage{graphicx}
\usepackage{graphics}
\usepackage{amsmath}
\usepackage{amsfonts}
\usepackage{amssymb}
\usepackage{amsthm}
\usepackage{xcolor}
\setlength{\oddsidemargin}{0.25 in}
\setlength{\evensidemargin}{-0.25 in}
\setlength{\topmargin}{-0.6 in}
\setlength{\textwidth}{6.5 in}
\setlength{\textheight}{8.5 in}
\setlength{\headsep}{0.75 in}
\setlength{\parindent}{0 in}
\setlength{\parskip}{0.1 in}

%
% The following commands set up the lecnum (lecture number)
% counter and make various numbering schemes work relative
% to the lecture number.
%
\newcounter{lecnum}
\renewcommand{\thepage}{\thelecnum-\arabic{page}}
\renewcommand{\thesection}{\thelecnum.\arabic{section}}
\renewcommand{\theequation}{\thelecnum.\arabic{equation}}
\renewcommand{\thefigure}{\thelecnum.\arabic{figure}}
\renewcommand{\thetable}{\thelecnum.\arabic{table}}

%
% The following macro is used to generate the header.
%
\newcommand{\lecture}[4]{
%   \pagestyle{myheadings}
   \thispagestyle{plain}
   \newpage
   \setcounter{lecnum}{#1}
   \setcounter{page}{1}
   \noindent
   \begin{center}
   \framebox{
      \vbox{\vspace{2mm}
    \hbox to 6.28in { {\bf CS 419M Introduction to Machine Learning
                        \hfill Spring 2021-22} }
       \vspace{4mm}
       \hbox to 6.28in { {\Large \hfill Lecture #1: #2  \hfill} }
       \vspace{2mm}
       \hbox to 6.28in { {\it Lecturer: #3 \hfill Scribe: #4} }
      \vspace{2mm}}
   }
   \end{center}
   \markboth{Lecture #1: #2}{Lecture #1: #2}
}

%
% Convention for citations is authors' initials followed by the year.
% For example, to cite a paper by Leighton and Maggs you would type
% \cite{LM89}, and to cite a paper by Strassen you would type \cite{S69}.
% (To avoid bibliography problems, for now we redefine the \cite command.)
% Also commands that create a suitable format for the reference list.
% \renewcommand{\cite}[1]{[#1]}
% \def\beginrefs{\begin{list}%
%         {[\arabic{equation}]}{\usecounter{equation}
%          \setlength{\leftmargin}{2.0truecm}\setlength{\labelsep}{0.4truecm}%
%          \setlength{\labelwidth}{1.6truecm}}}
% \def\endrefs{\end{list}}
% \def\bibentry#1{\item[\hbox{[#1]}]}

%Use this command for a figure; it puts a figure in wherever you want it.
%usage: \fig{NUMBER}{SPACE-IN-INCHES}{CAPTION}
% \newcommand{\fig}[3]{
% 			\vspace{#2}
% 			\begin{center}
% 			Figure \thelecnum.#1:~#3
% 			\end{center}
% 	}
% Use these for theorems, lemmas, proofs, etc.
\newtheorem{theorem}{Theorem}[lecnum]
\newtheorem{lemma}[theorem]{Lemma}
\newtheorem{proposition}[theorem]{Proposition}
\newtheorem{claim}[theorem]{Claim}
\newtheorem{corollary}[theorem]{Corollary}
\newtheorem{definition}[theorem]{Definition}
% \newenvironment{proof}{{\bf Proof:}}{\hfill\rule{2mm}{2mm}}

% **** IF YOU WANT TO DEFINE ADDITIONAL MACROS FOR YOURSELF, PUT THEM HERE:

\begin{document}
%FILL IN THE RIGHT INFO.
%\lecture{**LECTURE-NUMBER**}{**DATE**}{**LECTURER**}{**SCRIBE**}
\lecture{9}{Deriving stability of a classification}{Abir De}{Mahadevan Subramanian}
%\lecture{x}{Title}{Abir De}{Group y}

\section{Defining Stability}
Let us define sets of points $S$ to be consisting of points from $\mathbb{R}^n\times\{0,1\}$. $S$ is a data set where each point has some binary classification to it. Let us define the set of all these data sets as $\mathcal{S}$.\\
We define an algorithm $A:\mathcal{S}\to\mathbb{R}^d$ to be stable if $\textbf{Stab}_1(A)$ is a tautology.
\[\textbf{Stab}_1(A) = \forall{S}\in\mathcal{S},\forall{S'}\in\mathcal{S}\left[\left((|S\backslash S'|=|S'\backslash S|)\land(|S|=|S'|)\right)\to \left(\|A(S)-A(S')\| = \mathcal{O}\left(\dfrac{1}{|S|}\right)\right)\right]\]
The condition for this stability is for sets $S$ and $S'$ such that they differ only in element hence $S' = e'\cup(S\backslash e)$ where $e\neq e'$.\\
Similar to this condition let us define a new condition $\textbf{Stab}_2(A)$.
\[\textbf{Stab}_2(A) = \forall{S}\in\mathcal{S},\forall{e}\in S\left(\|A(S)-A(S\backslash e)\| = \mathcal{O}\left(\dfrac{1}{|S|}\right)\right)\]
\textbf{The overarching question:} For some $A$, do we have $\textbf{Stab}_1(A)\to\textbf{Stab}_2(A)$?
\begin{lemma}
For all algorithms $A:\mathcal{S}\to\mathbb{R}^d$, $\textbf{Stab}_2(A)\to\textbf{Stab}_1(A)$   
\end{lemma}
\begin{proof}
If we have $\textbf{Stab}_2(A)$ then $\|A(S)-A(S\backslash e)\| = \mathcal{O}(1/|S|)$ and $\|A(S')-A(S'\backslash e')\| = \mathcal{O}(1/|S'|)$.
\begin{align*}
    \|A(S) - A(S')\| &\leq \|A(S)-A(S\backslash e)\| + \|A(S'\backslash e') - A(S')\|\\
    &= \mathcal{O}(1/|S|) + \mathcal{O}(1/|S'|)\\
    &= \mathcal{O}\left(\frac{1}{|S|}\right)
\end{align*}
Hence we will have $\textbf{Stab}_1(A)$ hold given $\textbf{Stab}_2(A)$. 
\end{proof}

\section{Applying stability to classification}
Let us say we have a dataset $D = \{(x_i,y_i)\}$. Let us say we have some convex loss function $l(w^Tx,y)$ which is Lipschitz continuous. Let us define the following function over $S \subset D$ which has regularization
\[F_w(S) = \sum_S (l(w^Tx_i,y_i) + \lambda\|w\|^2)\]
Using this function we can define the following vector which minimizes the sum of the loss as
\[w^*(S) = \text{argmin}_wF_w(S)\]
\begin{proposition}
For the defined $F_w(S)$ with a convex and Lipschitz $l(w^Tx,y)$, $\textbf{Stab}_1(w^*)$ is true.
\end{proposition}
\begin{proof}
Let us define the notation $l(w^*(S),e) = l(w^*(S)^Tx,y)$. Now we take a close look at the value $F_{w^*(S')}(S) - F_{w^*(S)}(S)$. We must have the following hold
\[F_{w^*(S')}(S) - F_{w^*(S)}(S) = F_{w^*(S')}(S') - F_{w^*(S)}(S') + l(w^*(S'),e) - l(w^*(S),e) + l(w^*(S),e') - l(w^*(S'),e')\]
Since $w^*(S') = \text{argmin}_wF_w(S')$ we have $F_{w^*(S')}(S') - F_{w^*(S)}(S') \leq 0$ hence
\[F_{w^*(S')}(S) - F_{w^*(S)}(S) \leq l(w^*(S'),e) - l(w^*(S),e) + l(w^*(S),e') - l(w^*(S'),e') \leq 2L\|w^*(S)-w^*(S')\|\]
The last part of the inequality comes by combining the triangle inequality with the Lipschitz condition of $ l(w^*(S'),e) - l(w^*(S),e) \leq L\|w^*(S)-w^*(S')\|$.\\
We can also expand $F_{w^*(S')}(S) - F_{w^*(S)}(S)$ as a taylor expansion about the point $w^*(S)$.
\[F_{w^*(S')}(S) - F_{w^*(S)}(S) = \frac{\partial F_w(S)}{\partial w}\bigg|_{w = w^*(S)}(w - w^*(S)) + \frac{1}{2}(w-w^*(S))^TH(w-w^*(S)) + \dots\]
Here $H(F_w(S))$ is the Hessian for the function $F_w(S)$ with respect to $w$. We know that $w^*(S)$ minimizes $F_w(S)$ hence the first term vanishes and we are left with the inequality
\[F_{w^*(S')}(S) - F_{w^*(S)}(S) \geq \frac{1}{2}(w^*(S')-w^*(S))^TH(F_{w^*(S')}(S))(w^*(S')-w^*(S))\]
We know that $l(w,e)$ is a convex function hence the Hessian $H(l(w,e))$ is positive semi-definite. Hence we can surely conclude that the Hessian of the sum of all $l(w,e)$ terms is also positive semi-definite.\\
Now we can look at the regularization term, this will have to add a $2\lambda|S|I$ to the Hessian by definition and so we can conclude that $H(F_w(S)) \geq 2\lambda|S|I$ since the loss terms Hessian will anyways be positive semi-definite. Hence we have
\[F_{w^*(S')}(S) - F_{w^*(S)}(S) \geq \frac{2\lambda|S|}{2}(w^*(S')-w^*(S))^T(w^*(S')-w^*(S)) \geq \lambda|S|\|w^*(S')-w^*(S)\|^2\]
By combining the two inequalities we obtain by using first the Lipschitz condition and then that of convexity we obtain
\[\lambda|S|\|w^*(S')-w^*(S)\|^2 \leq F_{w^*(S')}(S) - F_{w^*(S)}(S) \leq 2L\|w^*(S)-w^*(S')\|\]
This subsequently reduces to 
\[\|w^*(S')-w^*(S)\| \leq \frac{2L}{\lambda|S|} = \mathcal{O}\left(\frac{1}{|S|}\right)\]
Hence we have proven that with a convex and Lipschitz $l(w^Tx,y)$, $\textbf{Stab}_1(w^*)$ is true.
\section{Group Details and Individual Contribution}
Group 1 for the scribe for lecture 9. All work in this copy done by Mahadevan Subramanian, Roll no. 190260027. 
\end{proof}
\end{document}





